%% Richard Wen
%% rwen@ryerson.ca

This proposal presents a research plan to develop open source Geographic Information System (GIS) software that monitors, models, and predicts road traffic crashes in near real time to support traffic-related decision making. Road traffic crashes take the lives of 3700 people every day -- resulting in 1.35 million lives lost each year \cite{who2018roadsafetyreport}. Over 50 million survivors of road injury suffer from disabilities, psychological trauma, financial losses, and legal burdens \cite{who2016postcrashresponse}. Road traffic crashes are often non-random events that can be prevented with changes to policy, infrastructure, health care, transportation, and culture -- reducing the severity of injuries and ultimately preventing the loss of human lives \cite{bonilla2014injuriesnotaccidents, who2017savelives}. In 1997, Sweden introduced a road safety policy called \textit{Vision Zero} that focuses on valuing human life, envisioning zero road traffic-related deaths, and sharing the responsibility between road users and designers \cite{belin2012visionzero}. Since the implementation of Vision Zero, the number of road traffic deaths have been reduced from 6 deaths per 100,000 people in 1997 to 4.7 deaths per 100,000 people in 2006 \cite{johansson2009visionzero}. In order to support decisions such as the implementation of Vision Zero, road traffic collisions need to be recorded as evidence and analyzed to better inform the decisions that change road traffic systems. These records have primarily been collected from police reports as data, which may not be available immediately or miss the context or characteristics surrounding each road traffic collision.
