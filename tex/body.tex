%% Richard Wen
%% rwen@ryerson.ca

This proposal presents a research plan to develop an open source Geographic Information System (GIS) software framework that supports and evaluates traffic-related decision making by monitoring, modeling, and predicting road traffic crashes. 

Across the world, road traffic crashes take the lives of 3700 people every day -- resulting in 1.35 million lives lost each year \cite{who2018roadsafetyreport}. Over 50 million survivors of road injury suffer from disabilities, psychological trauma, financial losses, and legal burdens \cite{who2016postcrashresponse}. Road traffic crashes are often non-random events that can be prevented with changes to policy, infrastructure, health care, transportation, and societal culture -- reducing the severity of injuries and ultimately preventing the loss of human lives over time \cite{bonilla2014injuriesnotaccidents,brubacher2014reduceroadtraumalaw,who2017savelives}. In 1997, Sweden introduced a road safety policy called \textit{Vision Zero} that centers road design around valuing human life, envisioning zero road traffic-related deaths, and sharing the responsibility between road users and designers \cite{belin2012visionzero}. Since the implementation of Vision Zero, the number of road traffic deaths were reduced from 6 deaths per 100,000 people in 1997 to 4.7 deaths per 100,000 people in 2006 \cite{johansson2009visionzero}. Road traffic collisions are recorded as data, analyzed, and then presented as evidence to support and evaluate decisions, such as the implementation of Vision Zero, that affect road traffic systems. These data have primarily been processed from police reports, which may not be immediately available, under-reported, or missing potentially useful information \cite{alsop2001underreportnz,loo2007reportroadcrash}. To account for the limitations in police reports, multiple sources of data for models and analyses can be used to cover a larger number of actual or potential road traffic crashes. These data sources can include hospital records \cite{cryer2001biashospitalpolicereports}, news reports \cite{dandona2004trafficdeathssurveillance}, real-time traffic data \cite{roshandel2015realtimecrashreview}, and, more recently, social media \cite{wanichayapong2011socialtrafficdata}. Although there has recently been an increase in data availability with the rapid growth of social media use \cite{perrin2015socialmediause} and efforts in making data open to the public, each data source is often inconsistent with each other and difficult to discover and use \cite{janssen2012opendata}. My research focuses on developing a GIS software framework that improves the compatibility of these data with each other, and reduces the technical knowledge required to use these data for monitoring, modeling, and predicting road traffic crashes -- enabling better evidence for traffic-related decisions by encompassing different perspectives with multiple sources of data, and developing comparable models with standardized software instructions known as code.

The design of the software framework will be based on the review of road traffic crash prediction models available in recent research literature from the the previous 5 to 10 years. Models are simplified representations of reality that can be created by computer programs for the purpose of predicting road traffic crashes \cite{rothenberg1989model}. These models are created with code that accepts different formats of data, processes the data, and discovers a pattern or a representation of the data that can be used to predict future road traffic crashes. Although this code is able to create models, it is often implemented differently among researchers, where the code may accept certain data inputs, process the data inconsistently, or produce predictions that are not comparable with other models. Software frameworks can solve these issues by providing standardized and comparable components containing the same code that can interact with each other and be extended with further functionality \cite{edwin2014softwareframeworks}. These components ensure that programs are consistent, but flexible enough to generate a variety of comparable road traffic crash prediction models, data inputs, and prediction outputs. The software framework for monitoring, modeling, and predicting road traffic crashes will be open sourced to promote transparent, comparable, and reproducible research. It will contain the following components:

\begin{enumerate}[label=\alph*)]
	\item Source: data sources for road traffic crashes
	\item Database: storage that can scale across several computers
	\item Model: road traffic crash prediction model with performance measures
	\item Prediction: predicted road traffic crashes
	\item Visualization: plots/maps showing visual information for the end user
\end{enumerate}

Currently, a prototype application has been created, a lab environment has been setup, and a
collaborative review paper has been published. The prototype web application visualizes real-time geo-located social media data from Twitter in Toronto, Ontario, with a dashboard interface. The interface monitors the collected social media data in real-time and presents useful count statistics that monitor the size and behavior of the data. A lab environment containing a scalable databases has been setup with graphical web interfaces to manage current and future road traffic crash related data. Future code will use these databases to conduct case studies, tests, and experiments for the proposed software framework. The published review paper surveys current methods of traffic event detection with geo-social media data to obtain an overview of the recent models, data sources, and visualization techniques used. This paper provides future directions and background knowledge for researchers to improve and develop geo-social event research in traffic applications. Another review paper will be written that further surveys road traffic crash prediction models that use geo-spatial data.

With the proposed GIS software framework and planned research for road traffic crashes, I hope to improve the evidence for supporting and justifying data-driven decisions that lead to the reduction in the number of road traffic related deaths and injuries.
